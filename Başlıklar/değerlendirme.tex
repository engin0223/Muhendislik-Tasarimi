\section{DEĞERLENDİRMELER}

Bu çalışmamızda, pasif bir devre elemanı olan LDR’ye derin öğrenme teknikleri uygulanarak sensör niteliği kazandırılmış ve düşük maliyetli, taşınabilir bir akıllı algılayıcı prototipi geliştirilmiştir. Kamera gibi yüksek işlem gücü ve veri aktarım kapasitesi gerektirmeyen bu sistem, özellikle nesne veya insan tanıma uygulamalarında kullanılabilecek alternatif bir çözüm sunmaktadır.

Cihazımız düşük bütçeli gömülü sistem projeleri, eğitim amaçlı uygulamalar ve IoT tabanlı akıllı cihazlar için uygun bir seçenek olup, bir bina girişine entegre edilerek kişi trafiğini izleyebilir. Trafik yoğunluğu, geçiş hızları veya sıra dışı davranış örüntülerine göre olası bir acil durumu tespit edebilen bu sistem, enerji verimli ve ekonomik bir IoT cihazı olarak güvenlik ve izleme sistemlerine entegre edilebilir.

Proje süresince bütçe dostu bileşenler tercih edilmiş; düşük maliyetli ve taşınabilir yapısı sayesinde saha uygulamalarına da elverişli bir sistem ortaya konmuştur. Geliştirme süreci boyunca ekip çalışmasının ve takım ruhunun önemi bir kez daha anlaşılmış, farklı alanlarda bilgi sahibi bireylerin katkılarıyla özgün ve yaratıcı bir ürün ortaya çıkarılmıştır. Teknik zorluklara ve zaman kısıtlarına rağmen ekip olarak organize biçimde çalışarak çözüm üretme kabiliyetimiz artmış, iletişim, görev paylaşımı ve problem çözme becerilerimiz pekişmiştir. Bu süreçte sıradan bir devre elemanına yenilikçi bir işlev kazandırarak mühendislik perspektifimizi genişlettik.

Sonuç olarak bu proje, yalnızca teknik anlamda değil, aynı zamanda disiplinler arası iş birliği, yaratıcı düşünme ve gerçek dünya problemlerine pratik çözümler geliştirme açısından da önemli kazanımlar sağlamıştır. Geliştirilen sistem, bir ekip emeğinin, stratejik planlamanın ve yenilikçi yaklaşımın somut bir ürünüdür. Gelecek çalışmalarda daha gelişmiş derin öğrenme modelleriyle doğruluk artırılabilir, farklı pasif devre elemanları da benzer yöntemlerle değerlendirilerek yeni nesil akıllı sensör çözümleri oluşturulabilir.

\begin{comment}
\textbf{8. KAYNAKLAR}

Tez kitabının ana gövdesi kaynaklar listesi ile son bulur. Kaynaklar
Tasarım/Bitirme Kitabı Yazım Klavuzunda açıklanan kurallara göre
yazılır. Nu kurallara göre;

\begin{enumerate}
\def\labelenumi{\arabic{enumi}.}
\item
  Yazarların ilk ve orta adları kısaltılıp, soyadları açık yazılır.
  Sadece ilk harfler büyük harfle yazılır.
\item
  Yazar adları sıralandıkdan sonra virgül konulup, tırnak içinde ilgili
  makale, kitap veya yazının başlığı yazılır.
\end{enumerate}

Başlıktan sonra kaynağın türüne göre aşağıdaki yazım kurallarına uyulur.

\begin{enumerate}
\def\labelenumi{\arabic{enumi}.}
\setcounter{enumi}{2}
\item
  \begin{quote}
  Sözkonusu kaynak dergi ise, başlıktan sonra virgül konulup, bu makale
  ve yazının yayınlandığı derginin adı, sayısı, bölüm numarası, yayın
  yılı ve makalenin yer aldığı başlangıç ve bitiş sayfalarının
  numaraları yazılır.
  \end{quote}
\item
  \begin{quote}
  Sözkonusu kaynak sempozyum veya konferans ise, başlıktan sonra virgül
  konulup, bu makale ve yazının yayınlandığı sempozyum veya konferansın
  adı yazılır. Sonra düzenlendiği yıl ve yer ile yayın yılı ve makalenin
  yer aldığı başlangıç ve bitiş sayfalarının numaraları yazılır.
  \end{quote}
\item
  \begin{quote}
  Sözkonusu kaynak kitap ise yayınlayan yayınevinin adı, kitabın basım
  yılı ve kaçıncı baskı olduğu bilgisi verilir.
  \end{quote}
\item
  \begin{quote}
  Sözkonusu kaynak tez ise, başlıktan sonra virgül konulup, bu tezin ne
  tezi olduğu (Bitirme Projesi, Yüksek Lisans tezi veya Doktora Tezi)
  bilgisi verilir. Tezin yapıldığı ünivbersite ve bölümünün adı verilir.
  Tezin yayın yılı yazılır.
  \end{quote}
\item
  \begin{quote}
  Web sayfasından alıntı yapılmışsa, web sayfasının adı ve çalışan
  bağlantı adresi verilir.
  \end{quote}
\end{enumerate}

\textbf{Örnekler:}

\textbf{Yazarlı Kitap}

\begin{enumerate}
\def\labelenumi{\arabic{enumi}.}
\item
  M. Buresch, ``\emph{Photovoltaic Energy Systems Design and
  Installation'',} McGraw-Hill, New York, 1983.
\item
  I. Boldea and Syed A. Nasar, "\emph{Linear Electric Actuators and
  Generators}", Cambridge University Press, 1997.
\end{enumerate}

\textbf{Editörlü Kitap}

\begin{enumerate}
\def\labelenumi{\arabic{enumi}.}
\setcounter{enumi}{2}
\item
  J. Breckling, Ed., ``\emph{The Analysis of Directional Time Series:
  Applications to Wind Speed and Direction''}, Lecture Notes in
  Statistics. Berlin, Germany: Springer, 1989, vol. 61.
\item
  A. A. Author1, B. B. Author2 and C. C. Author3, "Title of chapter or
  article", \emph{Name of the edited book}, A. A. Editor1 and B. B.
  Editor2 (Eds.), Publisher, Location, Year.
\end{enumerate}

\textbf{Dergi}

\begin{enumerate}
\def\labelenumi{\arabic{enumi}.}
\setcounter{enumi}{4}
\item
  L.A. Zadeh, "Fuzzy sets", \emph{Information and Control}, 8, 1965, pp.
  338-353.
\item
  W.Z. Fam and M.K. Balachander, "Dynamic Performance of a DC Shunt
  Motor Connected to a Photovoltaic Array", \emph{IEEE Trans. Energy
  Conversion, Vol. EC-3}, No.3, September 1988, pp.613-617.
\end{enumerate}

Yazar sayısı 3 den fazla ise:

\begin{enumerate}
\def\labelenumi{\arabic{enumi}.}
\setcounter{enumi}{6}
\item
  M. DeYong et al., "Fuzzy and adaptive control simulations for a
  walking machine", \emph{IEEE Control Systems}, Volume:12, Issue:3,
  June 1992, pp. 43-50.
\item
  A. Yazar ve diğerleri, ``Makalenin adı'', \emph{Derginin adı}, Varsa
  Bölüm Numarası, Sayısı, Basım tarihi, Sayfalar: 65-72.
\end{enumerate}

\textbf{Sempozyum veya Konferans}

\begin{enumerate}
\def\labelenumi{\arabic{enumi}.}
\setcounter{enumi}{8}
\item
  İ. H. Altaş, ``A Fuzzy Logic Controlled Tracking System For Moving
  Targets'', \emph{12\textsuperscript{th} IEEE International Symposium
  on Intelligent Control, ISIC'97}, July 16-18, 1997, Istanbul, Turkey,
  pp. 43-48.
\end{enumerate}

\textbf{Patent}

\begin{enumerate}
\def\labelenumi{\arabic{enumi}.}
\setcounter{enumi}{9}
\item
  R. E. Sorace, V. S. Reinhardt, and S. A. Vaughn, ``High-speed
  digital-to-RF converter,'' U.S. Patent 5 668 842, Sept. 16, 1997.
\end{enumerate}

\textbf{Web sayfası}

\begin{enumerate}
\def\labelenumi{\arabic{enumi}.}
\setcounter{enumi}{10}
\item
  International Energy Agency, ``Electricity and Heat for 2011'',
  website. [Online]. 
  (www.iea.org/statistics/statisticssearch/report/country=TURKEY=\&product=electricityandheat\&year=Select),
  Available as of June 22, 2014.
\item
  E-Mevzuat, ``Elektrik İç Tesisleri Yönetmeliği'', Mevzuat Geliştirme
  ve Yayın Genel Müdürlüğü, Mevzuat bilgi Sistemi, Web {[}Online{]}.
\end{enumerate}

\begin{quote}
(http://www.mevzuat.gov.tr/Metin.Aspx?MevzuatKod=7.5.10391\&sourceXmlSearch=\&MevzuatIliski=0),
Erişim tarihi: 22 Haziran 2014.
\end{quote}

\textbf{Data Sheet (Veri Sayfası)}

\begin{enumerate}
\def\labelenumi{\arabic{enumi}.}
\setcounter{enumi}{12}
\item
  \emph{FLEXChip Signal Processor (MC68175/D)}, Motorola, 1996.
\item
  ``PDCA12-70 data sheet,'' Opto Speed SA, Mezzovico, Switzerland.
\end{enumerate}

\textbf{Tez}

\begin{enumerate}
\def\labelenumi{\arabic{enumi}.}
\setcounter{enumi}{14}
\item
  A. Karnik, ``Performance of TCP congestion control with rate feedback:
  TCP/ABR and rate adaptive TCP/IP,'' M. Eng. Thesis, Indian Institute
  of Science, Bangalore, India, Jan. 1999.
\end{enumerate}

\textbf{Teknik Rapor}

\begin{enumerate}
\def\labelenumi{\arabic{enumi}.}
\setcounter{enumi}{15}
\item
  J. Padhye, V. Firoiu, and D. Towsley, ``A stochastic model of TCP Reno
  congestion avoidance and control,'' Univ. of Massachusetts, Amherst,
  MA, CMPSCI Tech. Rep. 99-02, 1999.
\end{enumerate}

\textbf{Standart}

\begin{enumerate}
\def\labelenumi{\arabic{enumi}.}
\setcounter{enumi}{16}
\item
  \emph{Wireless LAN Medium Access Control (MAC) and Physical Layer
  (PHY) Specification}, IEEE Std. 802.11, 1997.
\end{enumerate}

\textbf{EKLER}

Bitirme kitabında çalışmayla ilgili etik formlar, veri sayfaları, ürün
açıklaması, yazılım listesi ve teori detayı gibi açıklamalar ekler
bölümünde verilir.

Konulması gereken başlıca ekler aşağıda sıralanmıştır.

\begin{enumerate}
\def\labelenumi{\arabic{enumi}.}
\item
  IEEE Etik Kuralları (Türkçe) ve IEEE Code of Ethics (İngilizce)
\item
  Kısıtlar Formu
\end{enumerate}

\begin{quote}
Bu formda, çalışmayla ilgili gerçekleştirme ve uygulama kısıtları,
evrensel ve toplumsal boyutlarda sağlık, çevre ve güvenlik üzerindeki
etkileri hukuksal sonuçları sıralanmalıdır.
\end{quote}

\begin{enumerate}
\def\labelenumi{\arabic{enumi}.}
\setcounter{enumi}{2}
\item
  Disiplinlerarası Çalışma
\end{enumerate}

\begin{quote}
Bölüm Başkanlığı tarafından organize edilen ve katılma zorunluluğu
bulunan disiplşinlerarası çalıştaylarda veya derslerde edinilen
tecrübelere burada yer verilip anlatılmalıdır. \textbf{Bu kısım
zorunludur. Sözkonusu çalışmalara katılmayanların projesi kabul
edilmez.}

Ayrıca tasarım/Bitirme çalışmaları sırasında bölüm dışında başkalarına
yaptırılan veya başkaları ile birlikte çalışarak yapılan faaliyetlerin
nasıl yapıldığı ve yaptırıldığı anlatılmalıdır. Sözkonusu bölüm dışı
çalışmalara ne kadar süre ayrıldığı ve iletişim kurulan kişilerin
meslekleri hakkıında bilgi verilmelidir.
\end{quote}

\begin{enumerate}
\def\labelenumi{\arabic{enumi}.}
\setcounter{enumi}{3}
\item
  Diğer Ekler
\end{enumerate}

\begin{quote}
Veri sayfaları, ürün açıklamaları, yazılım listesi ve teori detayı gibi
ekler 5. Ekten başlanarak sıralanır. Araya başka ekler eklendikçe
aşağıdaki ek numaralarının da güncellenmesi gerekir.
\end{quote}

\begin{enumerate}
\def\labelenumi{\arabic{enumi}.}
\setcounter{enumi}{4}
\item
  Özgeçmişler
\item
  Mühendislik Tasarımı Teslim Koşulları Formu
\end{enumerate}

\begin{quote}
Bu form üzerinde gerekli işaretlemeler yapıldıktan sonra imzalanıp
taranarak eklenir. Bu formdaki bütütn soruları EVET olarak
cevaplayamayanlar Mühendislik Tasarımını teslim edemezler. Teslim
koşulları sağlanırsa form Mühendislik Tasarımı kitabına eklenir. Bitirme
projesi yazıldıktan sonra da olduğu yerde kalır.
\end{quote}

\begin{enumerate}
\def\labelenumi{\arabic{enumi}.}
\setcounter{enumi}{6}
\item
  Bitirme Projesi Teslim Koşulları Formu
\end{enumerate}

\begin{quote}
Bu form üzerinde gerekli işaretlemeler yapıldıktan sonra imzalanıp
taranarak Bitirme Projesi kitabına eklenir. Bu formdaki bütütn soruları
EVET olarak cevaplayamayanlar Bitirme Projesiini teslim edemezler.
Teslim koşulları sağlanırsa bu form Bitirme Projesi kitabına Mühendislik
Tasarımı Koşullarını Sağlama Formundan sonra eklenir. Mühendislik
Tasarımı aşamasında bu form eklenmez.
\end{quote}

\begin{enumerate}
\def\labelenumi{\arabic{enumi}.}
\setcounter{enumi}{7}
\item
  TÜBİTAK proje kapatma formu
\end{enumerate}

\begin{quote}
TÜBİTAK tarafından desteklenen 2209/B vb projeler tamamlandıktan sonra
TÜBİTAK proje sayfasından alacakları kapatma formunun danışman ve
öğrenciler tarafından imzalı hali son ek olarak sadece Bitirme Projesine
eklenmelidir.
\end{quote}
\end{comment}

