{\centering \textbf{ÖZET}\par}

{
\setlength{\parindent}{0pt}
Nesnelerin İnterneti (IoT), bilgisayar olarak kabul edilmeyen nesnelerin ve sensörlerin ağ bağlantısı aracılığıyla veri üretmesi, paylaşması ve işlemesine olanak tanıyan bir konsepttir.  

Nesnelerin İnterneti cihazları, bilgi üretme ve paylaşma yetenekleri sayesinde büyük veri alanında önemli bir kaynak oluşturmaktadır. Nesnelerin İnterneti kavramına uyan cihaz sayısı her geçen gün hızla artmakta ve bu cihazların oluşturduğu verilerle beraber bu verilerin mümkün kullanım alanları da gelişmektedir. 


Sinir ağları, büyük ve karmaşık veri kümelerinden anlamlı bilgiler çıkarmada etkili araçlar sunar. Sinir Ağı teknolojisi, görüntü tanıma, ses işleme ve metin analizi gibi çeşitli alanlarda başarı göstermiştir. IoT tabanlı sistemlerde derin öğrenme, sensörlerden toplanan verilerin gerçek zamanlı olarak analiz edilmesini sağlayarak daha akıllı ve özelleştirilebilir çözümler sunmaktadır. Derin öğrenme teknolojilerinin gelişimi, artan işlem gücü ve veri erişilebilirliği ile hız kazanmış, IoT uygulamalarını daha verimli ve ölçeklenebilir hale getirmiştir. 

Bu çalışma, ışığa bağımlı dirençten (LDR) elde edilen veriyi sinir ağları teknolojisinin bir türü olan derin öğrenme ağları yöntemiyle analiz ederek kişi tespiti yapılması ve hem elde edilen verinin hem de tespitin diğer cihazlar arasında iletimini mümkün kılan bir devre tasarımını gerçekleştirmektedir.  

Bu devre tasarımı ile oluşturulan devrenin binaların giriş noktalarına bir sensör olarak yerleştirilmesi ve internet bağlantısı ile bu sensörden elde edilen verinin diğer cihazlara aktarılması mümkün kılınmaktadır.
}

\phantomsection
\addcontentsline{toc}{subsection}{\hspace{-1.5em} ÖZET }