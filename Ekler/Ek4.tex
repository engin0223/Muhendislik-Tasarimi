Disiplinler arası Çalışma formu iki kısımdan oluşmaktadır. Bunlardan birincisi Bölüm Başkanlığı tarafından organize edilen ve katılımın zorunlu olduğu disiplinler arası çalışmadır. İkincisi ise öğrenci grupları tarafından yapılan proje çalışmalarına veya Bitirme Projesi yapılırken bölüm dışında farklı meslekten kişilerle yapılan projeye destek çalışmalarıdır.


\begin{table}[H]
\centering
\begin{tabular}{|p{7cm}|p{7cm}|}
\hline
  Disiplinler arası çalışma &
  Açıklam \\ \hline
  Bölüm Başkanlığınca organize edilen “Girişimcilik” çalıştayı &
  
  Engin KAYA ve Taha Ramazan UYSAL: Bu proje, toplu taşıma hatlarındaki yoğunluğu gerçek zamanlı izleyerek otobüs seferlerini optimize etmeyi amaçlar. Yolcu yoğunluğu verileri analiz edilerek, yoğun hatlarda ek sefer düzenlemesi ve düşük yoğunluklu hatlarda kaynak tasarrufu sağlanır.

  İdrishan KANBER: Bu proje, doğal afetler sırasında mahsur kalanların yerini tespit etmek için termal kamera ve düşük frekanslı radyo dalgalarını kullanan bir telsiz tasarımı sunar. Radyo dalgaları yoğun malzemelerden geçerken, termal kameralar canlıları tespit eder, böylece ilk yardım ekiplerinin hızlı müdahalesi sağlanır.
  
  \\ \hline
\end{tabular}
\end{table}